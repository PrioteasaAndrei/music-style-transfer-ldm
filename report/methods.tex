% \section{Methods}

% \subsection{Dataset}
% The project uses a custom dataset for training and evaluation:
% \begin{itemize}
%     \item Spectrograms are processed to a fixed size of 128x128
%     \item Data is organized in pairs of content and style spectrograms
%     \item The dataset includes various musical styles and instruments
%     \item Spectrograms are normalized to the range [-1, 1]
% \end{itemize}

% \subsection{Model Architecture}
% The complete model architecture consists of several interconnected components:

% \subsubsection{Latent Diffusion Model}
% The main LDM class integrates all components:
% \begin{lstlisting}
% class LDM(nn.Module):
%     def __init__(self, latent_dim, num_timesteps):
%         self.encoder = SpectrogramEncoder(latent_dim)
%         self.decoder = SpectrogramDecoder(latent_dim)
%         self.style_encoder = StyleEncoder()
%         self.forward_diffusion = ForwardDiffusion(num_timesteps)
%         self.unet = UNet()
% \end{lstlisting}

% \subsubsection{UNet Architecture}
% The UNet is designed to handle the diffusion process:
% \begin{lstlisting}
% class UNet(nn.Module):
%     def __init__(self, in_channels=1, out_channels=1, num_filters=64):
%         # Encoder path
%         self.enc1 = nn.Conv2d(in_channels, num_filters, 3, padding=1)
%         self.enc2 = nn.Conv2d(num_filters, num_filters*2, 3, stride=2, padding=1)
%         self.enc3 = nn.Conv2d(num_filters*2, num_filters*4, 3, stride=2, padding=1)
%         self.enc4 = nn.Conv2d(num_filters*4, num_filters*8, 3, stride=2, padding=1)
        
%         # Decoder path
%         self.dec4 = nn.ConvTranspose2d(num_filters*8, num_filters*4, 4, stride=2, padding=1)
%         self.dec3 = nn.ConvTranspose2d(num_filters*4, num_filters*2, 4, stride=2, padding=1)
%         self.dec2 = nn.ConvTranspose2d(num_filters*2, num_filters, 4, stride=2, padding=1)
%         self.dec1 = nn.Conv2d(num_filters, out_channels, 1)
% \end{lstlisting}

% \subsection{Training Pipeline}
% The training process is implemented using PyTorch Lightning for efficient training:

% \begin{enumerate}
%     \item \textbf{Data Loading}:
%     \begin{itemize}
%         \item Custom DataLoader for spectrogram pairs
%         \item Batch processing with appropriate normalization
%         \item Data augmentation techniques
%     \end{itemize}
    
%     \item \textbf{Training Loop}:
%     \begin{itemize}
%         \item Two-phase training: autoencoder and style transfer
%         \item Gradient clipping and learning rate scheduling
%         \item Checkpointing and model saving
%     \end{itemize}
    
%     \item \textbf{Inference}:
%     \begin{itemize}
%         \item DDIM sampling for faster generation
%         \item Style conditioning during inference
%         \item Post-processing of generated spectrograms
%     \end{itemize}
% \end{enumerate}

% \subsection{Evaluation Metrics}
% The model's performance is evaluated using several metrics:

% \begin{itemize}
%     \item \textbf{Reconstruction Quality}:
%     \begin{itemize}
%         \item Mean Squared Error (MSE)
%         \item Perceptual loss using VGGish features
%         \item KL divergence in latent space
%     \end{itemize}
    
%     \item \textbf{Style Transfer Quality}:
%     \begin{itemize}
%         \item Style loss between target and generated spectrograms
%         \item Content preservation metrics
%         \item Qualitative evaluation through audio samples
%     \end{itemize}
    
%     \item \textbf{Computational Efficiency}:
%     \begin{itemize}
%         \item Training time per epoch
%         \item Inference time for style transfer
%         \item Memory usage during training and inference
%     \end{itemize}
% \end{itemize}

% The implementation includes various optimizations and best practices:
% \begin{itemize}
%     \item Efficient data loading and preprocessing
%     \item Gradient checkpointing for memory efficiency
%     \item Mixed precision training
%     \item Distributed training support
%     \item Comprehensive logging and visualization
% \end{itemize} 