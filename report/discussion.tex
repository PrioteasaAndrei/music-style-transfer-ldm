\section{Discussion}

\subsection{Challenges}
During the development of this project, several significant challenges were encountered:

\begin{itemize}
    \item \textbf{Data Processing}:
    \begin{itemize}
        \item Ensuring consistent spectrogram quality and normalization
        \item Handling different audio lengths and sampling rates
        \item Creating balanced pairs of content and style spectrograms
    \end{itemize}
    
    \item \textbf{Model Architecture}:
    \begin{itemize}
        \item Balancing the trade-off between compression ratio and quality
        \item Designing effective style conditioning mechanisms
        \item Optimizing the UNet architecture for spectrogram processing
    \end{itemize}
    
    \item \textbf{Training Process}:
    \begin{itemize}
        \item Managing multiple loss components and their weights
        \item Achieving stable training with the diffusion process
        \item Handling memory constraints during training
    \end{itemize}
\end{itemize}

\subsection{Limitations}
The current implementation has several limitations that could be addressed in future work:

\begin{itemize}
    \item \textbf{Technical Limitations}:
    \begin{itemize}
        \item Fixed spectrogram size (128x128) may not capture all musical details
        \item Limited to monophonic audio processing
        \item Requires paired training data
    \end{itemize}
    
    \item \textbf{Musical Limitations}:
    \begin{itemize}
        \item Difficulty in preserving complex polyphonic structures
        \item Limited ability to transfer expressive musical elements
        \item Challenges with maintaining musical coherence in longer pieces
    \end{itemize}
    
    \item \textbf{Computational Limitations}:
    \begin{itemize}
        \item Training time could be reduced with better optimization
        \item Memory usage could be optimized for consumer hardware
        \item Real-time processing is not yet achieved
    \end{itemize}
\end{itemize}

\subsection{Future Work}
Several promising directions for future research and development:

\begin{itemize}
    \item \textbf{Architectural Improvements}:
    \begin{itemize}
        \item Implement hierarchical latent spaces for better feature extraction
        \item Develop attention mechanisms for better style conditioning
        \item Explore transformer-based architectures for sequence modeling
    \end{itemize}
    
    \item \textbf{Training Enhancements}:
    \begin{itemize}
        \item Develop self-supervised learning approaches
        \item Implement curriculum learning for better convergence
        \item Create more sophisticated loss functions
    \end{itemize}
    
    \item \textbf{Feature Extensions}:
    \begin{itemize}
        \item Support for polyphonic music
        \item Real-time processing capabilities
        \item Integration with other audio processing tasks
    \end{itemize}
    
    \item \textbf{Applications}:
    \begin{itemize}
        \item Music production and composition tools
        \item Educational applications for music theory
        \item Audio restoration and enhancement
    \end{itemize}
\end{itemize}

\subsection{Impact and Implications}
The project's findings have several important implications:

\begin{itemize}
    \item \textbf{Technical Impact}:
    \begin{itemize}
        \item Demonstrates the effectiveness of LDMs for audio processing
        \item Provides insights into spectrogram-based style transfer
        \item Offers a framework for future audio processing research
    \end{itemize}
    
    \item \textbf{Practical Impact}:
    \begin{itemize}
        \item Potential for music production and education
        \item Applications in audio restoration and enhancement
        \item Tools for music analysis and understanding
    \end{itemize}
    
    \item \textbf{Research Impact}:
    \begin{itemize}
        \item Advances in audio style transfer methodology
        \item Insights into latent space representations of music
        \item Contributions to the field of audio processing
    \end{itemize}
\end{itemize} 