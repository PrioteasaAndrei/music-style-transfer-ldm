\section{Introduction}

\subsection{Background}
Music style transfer involves altering the style of a musical piece, 
such as changing the instrument its played with, 
while preserving original characteristics like melody or rhythm.
This could be done by relying on rule-based systems, 
by e\.g\. modifying the instruments in an MIDI file inside a music production software, 
or by leveraging machine learning techniques for symbolic music genre transfer, like Brunner et al\. with MIDI-VAE or CycleGAN~\cite{Brunner2018MIDIVAEMD, Brunner2018SymbolicMG}.
However these approaches are limited to leveraging midi files with the latter also requiring an exceptional amount of compute to train.


However capturing the essence of a musical piece and transferring it to another one can involve much more than just 
changing what instrument is used to play the notes, if even applicable.

Recent advancements in machine learning have introduced approaches that work very well for the same corresponding task in 
the image domain. Notably, Latent Diffusion Models (LDMs) have demonstrated remarkable success 
in more efficiently generating high-quality images~\cite{rombach2022high}, and these models can be adapted to effectively transfer styles.
% write some more sentences about ldms (whats the OG paper, we should cite it here)

The principles of LDMs can be extended to music by representing audio data as spectrograms, 
visual representations of the frequencies in a sound signal over time.
This approach potentially allows for application of image-based generation and style transfer techniques to audio data.
The paper `Music Style Transfer With Diffusion Model'~\cite{huang2024music} presents exactly this approach. 
Proposing a framework that utilizes diffusion models for music style transfer.
% two more detailed sentences over the paper

In this project, we want to implement a version of the approach proposed in the paper.
However, since their code is not publicly available, we will develop our own implementation and hope to achieve similar results.

\subsection{Problem Statement \& Project Goals \& Challenges}
The goal of this project is to implement a music style transfer architecture based on the principles of Latent Diffusion Models (LDMs).
We should be able to take a music sample and change its instrument while preserving its other characteristics.
This involves several challenges, including:

\begin{itemize}
    \item \textbf{Dataset:} Creating a suitable dataset for training the model. 
      The dataset should contain music samples with different instruments. Having enough samples of each instrument is also important to 
      model has enough data to learn from. This is important to ensure that the model can learn to transfer styles effectively. 
      Since we are very constrained in terms of computing resources, we restrict ourselves to only a few instruments and small samples.

    \item \textbf{Model Architecture:} Designing the model architecture. 
      We will be implementing a Latent Diffusion Model (LDM) as proposed in the paper.
      Implementing an LDM as it would be traditionally done for images and also incorporating the ideas from the paper to adapt it to music.
      % Add some specifics?

    \item \textbf{Hardware Limitations:}
      Since we only have access to our laptops and a single 3060ti GPU, we are very limited in terms of computing resources.
      Because of that we have to be very careful to not introduce too many parameters to the model in order to train it for enough epochs.
      
    \item \textbf{Quality of Generated Samples:}
      We want to generate music samples that still sound okay and let us hear the characteristics behind the song and instruments.
      We plan to do most of the evacuation concerning the quality of the generated samples by just listening to them.
      However since we are very restricted in terms of computing resources, we are totally fine with not archiving the highest quality.
      If we can generate samples that sound okay and are able to transfer the style of the music, this is already a success and validate our approach
      With more data, larger models and more training, someone with more resources could then probably achieve better results.
\end{itemize}