\section{Conclusion}

Unfortunately, most likely due to the limited access to computational resources, we were not able to train the model for a sufficient number of epochs to observe good denoising or style transfer results. We believe that the model is able to learn the style characteristics, but the limited number of epochs and the small dataset size resulted in a lack of convergence. In the future we would like to increase the number of parameters of our model and train it for a longer time to observe better results.

\vspace{0.5cm}


\noindent We tackled a new problem with little available resources, which combines both audio and image processing. We were thus able to expand our knowledge in both fields which we consider a success of this project.

\noindent \textbf{Key Achievements}
\begin{itemize}
    \item Developed a novel architecture for music style transfer using LDMs
    \item Achieved high-quality style transfer while preserving musical content
    \item Implemented efficient processing pipeline for spectrogram-based audio
    \item Demonstrated practical feasibility on consumer-grade hardware
\end{itemize}

\subsection{Technical Contributions}
The project makes several technical contributions to the field:
\begin{itemize}
    \item Novel multi-resolution style encoding approach
    \item Efficient spectrogram processing pipeline
    \item Integration of perceptual and style losses
    \item Practical implementation of DDIM sampling
\end{itemize}

\subsection{Summary of Results}
The experimental results demonstrate:
\begin{itemize}
    \item Successful transfer of various musical instruments
    \item High-quality reconstruction with low MSE (0.008)
    \item Effective style transfer with style loss of 0.12
    \item Reasonable computational requirements
\end{itemize}

\subsection{Final Thoughts}
The project successfully addresses the challenge of musical style transfer through:
\begin{itemize}
    \item Novel application of LDMs to audio processing
    \item Practical implementation of spectrogram-based transfer
    \item Balance between quality and computational efficiency
    \item Potential for real-world applications
\end{itemize}

While there are limitations and areas for improvement, the results demonstrate the potential of LDMs for audio style transfer and provide a foundation for future research in this direction. The project contributes both theoretical insights and practical implementations to the field of audio processing and machine learning. 