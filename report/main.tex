\documentclass[12pt]{article}
\usepackage[utf8]{inputenc}
\usepackage{graphicx}
\usepackage{amsmath}
\usepackage{amssymb}
\usepackage{hyperref}
\usepackage{listings}
\usepackage{xcolor}
\usepackage{booktabs}
\usepackage{multirow}
\usepackage{subcaption}

% Code listing settings
\lstset{
    language=Python,
    basicstyle=\ttfamily\tiny,
    breaklines=true,
    keywordstyle=\color{blue},
    stringstyle=\color{red},
    commentstyle=\color{green!60!black},
    numbers=left,
    numberstyle=\tiny,
    numbersep=5pt,
    frame=single,
    showstringspaces=false
}

\title{Musical Style Transfer Using Latent Diffusion Models}
\author{Andrei Prioteasa (Matr. No.: 4740844) \and Theo Stempel-Hauburger (Matr. No.: 4740729)}
\date{\today}

\newcommand{\lecturename}{Lecture: Generative Neural Networks for the Sciences, WS 2024/25}
\newcommand{\lecturer}{Lecturer: Prof. Dr. Ullrich Köthe}
\newcommand{\github}{GitHub Repository: \url{https://github.com/PrioteasaAndrei/music-style-transfer-ldm.git}}

\begin{document}

\maketitle
\begin{center}
    \lecturename \\
    \lecturer \\
    \vspace{0.5cm}
    \github
\end{center}

\tableofcontents
\begin{abstract}
    \textit{Andrei P.}\\
    \noindent In this project we implement a version of the approach proposed in the paper `Music Style Transfer With Diffusion Model'~\cite{huang2024music}. We experiment with the use of latent diffusion models for musical style transfer, working on grayscale mel-spectrograms (which are a visual representation of audio). We design a Latent Diffusion Model that can transfer the style of one instrument to another, while preserving the original tonal characteristics of the instrument. We collect and preprocess our own dataset from youtube videos, which we make available to the public (see \url{https://github.com/PrioteasaAndrei/music-style-transfer-ldm.git}). 
    
    We analyze the model capability of transferring the style of one instrument to another, generating new spectrograms conditioned on a style image and we run experiments on the architecture choices. We conclude that we are able to obtain blurry reconstruction, limited style transfer capabilities. We consider that the limited results of our model are due to the limited computational resources available to us and the complexity of the task, further training and refinement of our code is required to improve the results. We present a sample of our conditioned generation result at: \url{https://github.com/PrioteasaAndrei/music-style-transfer-ldm/blob/main/report/figures/content_aware_generated_audio_200ep.mp3}
   
\end{abstract}

\section{Introduction}

\subsection{Background}
Music style transfer is a challenging task in the field of audio processing and machine learning. The goal is to transform a piece of music from one style to another while preserving its content and musical structure. Traditional approaches to style transfer have often relied on rule-based systems or simple signal processing techniques, which have limited capabilities in capturing complex musical styles and maintaining musical coherence.

Recent advances in deep learning, particularly in the field of generative models, have opened new possibilities for music style transfer. Latent Diffusion Models (LDMs) have shown remarkable success in image generation and style transfer tasks, offering a promising approach for music processing. By operating in a compressed latent space, LDMs can efficiently capture and manipulate high-level features while maintaining computational efficiency.

\subsection{Problem Statement}
The main challenges in music style transfer include:
\begin{itemize}
    \item Preserving the musical content while changing the style
    \item Maintaining temporal coherence and musical structure
    \item Handling the high-dimensional nature of audio data
    \item Ensuring real-time processing capabilities
    \item Achieving high-quality results with limited computational resources
\end{itemize}

Traditional methods often struggle with these challenges, particularly in maintaining musical coherence and handling complex style transformations. The need for a more robust and efficient approach has led to the exploration of latent diffusion models for this task.

\subsection{Project Goals}
This project aims to:
\begin{itemize}
    \item Implement a novel approach to music style transfer using latent diffusion models
    \item Develop an efficient architecture that can process spectrograms in real-time
    \item Create a system that can transfer musical styles while preserving content
    \item Evaluate the effectiveness of different loss functions and training strategies
    \item Provide a practical solution that can run on consumer-grade hardware
\end{itemize}

The implementation focuses on spectrogram-based processing, which allows for efficient handling of audio data while maintaining the temporal and frequency characteristics of the music. By leveraging the power of latent diffusion models, we aim to achieve high-quality style transfer results while addressing the computational challenges associated with audio processing. 
\section{Methodology}

In this section, we will describe the methodology used to implement the proposed method. We will describe our approach to the architecture, the main components and the training process.

\subsection{Architecture Overview}
We tried to stay as close to the original paper as possible in terms of architecture, but the paper did not provide a detailed description. The main components of our model are:

\begin{table}[h]
\centering
\begin{tabular}{|l|p{10cm}|}
\hline
\textbf{Component} & \textbf{Description} \\
\hline
Spectrogram Encoder & Compresses the input spectrogram into a latent space \\
\hline
Style Encoder & Processes style spectrograms to extract multi-resolution style embeddings \\
\hline
Forward Diffusion & Implements the noise scheduler \\
\hline
UNet & Denoises the latent representation \\
\hline
Spectrogram Decoder & Reconstructs the final spectrogram from the latent space \\
\hline
DDIM & Reverse sampling process for generating new samples \\
\hline
Cross-Attention & Adds style information to the denoising process \\
\hline
VGGishFeatureLoss & Pretrained VGGish model to extract features from the spectrogram \\
\hline
\end{tabular}
\caption{Main components of the model architecture}
\label{tab:model-components}
\end{table}

\noindent We will now briefly describe each of the components.

\subsubsection{Spectrogram Encoder and decoder}
The encoder compresses the input spectrogram into a latent space using a series of convolutional layers, which allows for unrestricted input size:
\begin{lstlisting}[basicstyle=\tiny]
class SpectrogramEncoder(nn.Module):
    def __init__(self, latent_dim=4):
        self.encoder = nn.Sequential(
            nn.Conv2d(1, 64, kernel_size=3, stride=2, padding=1),
            nn.BatchNorm2d(64),
            nn.ReLU(),
            nn.Conv2d(64, 128, kernel_size=3, stride=2, padding=1),
            nn.BatchNorm2d(128),
            nn.ReLU(),
            nn.Conv2d(128, latent_dim, kernel_size=3, stride=2, padding=1),
            nn.BatchNorm2d(latent_dim)
        )
\end{lstlisting}

\noindent  The decoder mirrors the encoder architecture but uses transposed convolutions to upsample back to the original dimensions, normalizing the output to be between -1 and 1:
\begin{lstlisting}[basicstyle=\tiny]
class SpectrogramDecoder(nn.Module):
    def __init__(self, latent_dim=4):
        self.decoder = nn.Sequential(
            nn.ConvTranspose2d(latent_dim, 128, kernel_size=3, stride=2, padding=1, output_padding=1),
            nn.BatchNorm2d(128),
            nn.ReLU(),
            nn.ConvTranspose2d(128, 64, kernel_size=3, stride=2, padding=1, output_padding=1),
            nn.BatchNorm2d(64),
            nn.ReLU(),
            nn.ConvTranspose2d(64, 1, kernel_size=3, stride=2, padding=1, output_padding=1),
            nn.Tanh()
        )
\end{lstlisting}

\noindent Both the encoder and decoder need to be pretrained on the spectrograms to be able to reconstruct the original audio. During the training process, we froze the encoder weights to prevent them from being updated, while leaving the decoder weights trainable. We describe this process in the experiments section.

\subsubsection{Style Encoder}
The style encoder processes style spectrograms to extract multi-resolution embeddings. Activation maps from different convolutional layers are extracted and used as conditioning mechianisms in the UNet, thorugh the Cross Attention mechanism.

\begin{lstlisting}[basicstyle=\tiny]
class StyleEncoder(nn.Module):
    def __init__(self, in_channels=1, num_filters=64):
        self.enc1 = nn.Conv2d(in_channels, num_filters, kernel_size=3, stride=2, padding=1)
        self.enc2 = nn.Conv2d(num_filters, num_filters * 2, kernel_size=3, stride=2, padding=1)
        self.enc3 = nn.Conv2d(num_filters * 2, num_filters * 4, kernel_size=3, stride=2, padding=1)
        self.enc4 = nn.Conv2d(num_filters * 4, num_filters * 4, kernel_size=3, stride=2, padding=1)
        self.enc5 = nn.Conv2d(num_filters * 4, num_filters * 4, kernel_size=3, stride=2, padding=1)
        self.enc6 = nn.Conv2d(num_filters * 4, num_filters * 8, kernel_size=3, stride=2, padding=1)
\end{lstlisting}

\subsubsection{UNet}
The UNet is a standard denoising diffusion model, which is used to denoise the latent representation of the spectrogram. The UNet is conditioned on the style embeddings extracted by the style encoder using the Cross Attention mechanism. Skip connections are used to improve the training process. We use a sinusoidal position embedding to encode the time step in the UNet.


\newpage

\begin{lstlisting}[basicstyle=\tiny]
class UNet(nn.Module):
    def __init__(self, in_channels=1, out_channels=1, num_filters=64):
        super(UNet, self).__init__()

        # Define the channel dimensions used in your UNet
        time_emb_dim = 128  # This should match the channel dimension where you add the time embedding
        
        self.time_mlp = nn.Sequential(
            SinusoidalPositionEmbeddings(time_emb_dim),  # Match the channel dimension
            nn.Linear(time_emb_dim, time_emb_dim),
            nn.GELU(),
            nn.Linear(time_emb_dim, time_emb_dim),
        )

        # Downsampling path with proper padding to maintain spatial dimensions
        self.enc1 = nn.Conv2d(in_channels, num_filters, kernel_size=3, stride=1, padding=1)
        self.enc2 = nn.Conv2d(num_filters, num_filters * 2, kernel_size=3, stride=2, padding=1)  # 128x128
        self.enc3 = nn.Conv2d(num_filters * 2, num_filters * 4, kernel_size=3, stride=2, padding=1)  # 64x64
        self.enc4 = nn.Conv2d(num_filters * 4, num_filters * 8, kernel_size=3, stride=2, padding=1)  # 32x32

        # Cross attention layers with correct embedding dimensions
        self.cross_attention1 = CrossAttention(embed_dim=512, num_heads=4)  # For 2x2 feature maps with 512 channels
        self.cross_attention2 = CrossAttention(embed_dim=256, num_heads=4)  # For 4x4 feature maps with 256 channels

        # Bottleneck
        self.bottleneck = nn.Conv2d(num_filters * 8, num_filters * 8, kernel_size=3, stride=1, padding=1)

        # Upsampling path with proper padding to maintain spatial dimensions
        self.dec4 = nn.ConvTranspose2d(num_filters * 8, num_filters * 4, kernel_size=3, stride=2, padding=1, output_padding=1)  # 64x64
        self.dec3 = nn.ConvTranspose2d(num_filters * 4, num_filters * 2, kernel_size=3, stride=2, padding=1, output_padding=1)  # 128x128
        self.dec2 = nn.ConvTranspose2d(num_filters * 2, num_filters, kernel_size=3, stride=2, padding=1, output_padding=1)  # 256x256
        self.dec1 = nn.Conv2d(num_filters, out_channels, kernel_size=3, stride=1, padding=1)

\end{lstlisting}

\subsubsection{ForwardDiffusion}

The forward diffusion process gradually adds Gaussian noise to the input data over a fixed number of timesteps. At each timestep $t$, the process is defined by:

\begin{equation}
    q(z_t|z_{t-1}) = \mathcal{N}(z_t; \sqrt{1-\beta_t}z_{t-1}, \beta_t\mathbf{I})
\end{equation}

\noindent where $\beta_t$ is a variance schedule that controls how much noise is added at each step. The process can be written in a closed form for any timestep $t$ as:

\begin{equation}
    q(z_t|z_0) = \mathcal{N}(z_t; \sqrt{\bar{\alpha}_t}z_0, (1-\bar{\alpha}_t)\mathbf{I})
\end{equation}

\noindent where $\alpha_t = 1-\beta_t$ and $\bar{\alpha}_t = \prod_{s=1}^t \alpha_s$. This allows us to sample $z_t$ directly for any timestep using the reparameterization trick:

\begin{equation}
    z_t = \sqrt{\bar{\alpha}_t}z_0 + \sqrt{1-\bar{\alpha}_t}\epsilon, \quad \epsilon \sim \mathcal{N}(0, \mathbf{I})
\end{equation}

The reverse process then learns to gradually denoise the data by predicting the noise $\epsilon$ at each timestep. Given a noisy sample $z_t$ and timestep $t$, we can predict the original input $z_0$ using:

\begin{equation}
    z_0 = \frac{z_t - \sqrt{1-\bar{\alpha}_t}\epsilon_\theta(z_t,t)}{\sqrt{\bar{\alpha}_t}}
\end{equation}

\noindent where $\epsilon_\theta$ is our UNet model that predicts the noise. This formulation allows for stable training and high-quality generation.

\subsubsection{DDIM Sampling}
\label{sec:ddim_sampling}
% \begin{equation}
%     q(z_t|z_{t-1}) = \mathcal{N}(z_t; \sqrt{1-\beta_t}\,z_{t-1},\, \beta_t\mathbf{I}),
% \end{equation}
% and in closed form,
% \begin{equation}
%     q(z_t|z_0) = \mathcal{N}(z_t; \sqrt{\bar{\alpha}_t}\,z_0,\, (1-\bar{\alpha}_t)\mathbf{I}),
% \end{equation}
% with
% \begin{equation}
%     z_t = \sqrt{\bar{\alpha}_t}\,z_0 + \sqrt{1-\bar{\alpha}_t}\,\epsilon, \quad \epsilon \sim \mathcal{N}(0, \mathbf{I}).
% \end{equation}

This process was originally introduced by Song et al.~\cite{song2021denoising}. 
Opposed to the forward diffusion process, which gradually adds Gaussian noise to the input data, in DDIM sampling, our goal is to reverse this process. 
\\\\
Given the noisy latent representation of our original image \( z_t \), 
we use a trained UNet model \(\epsilon_\theta(z_t,t)\) to predict the noise in \( z_t \). 
From this prediction, we first estimate the original latent variable \( z_0 \) by rearranging the forward process equation:
\begin{equation}
    z_0^{\text{pred}} = \frac{z_t - \sqrt{1-\bar{\alpha}_t}\,\epsilon_\theta(z_t,t)}{\sqrt{\bar{\alpha}_t}}.
\end{equation}
When we have a lot of noise, this is a pretty hard task for the model and therefore the \( z_0^{\text{pred}} \) will only be a rough estimate.
Once we have this estimate, we add back a portion of the noise to reconstruct \( z_{t-1} \), the latent at the previous timestep.
In other words, we again “mix” the estimated \( z_0 \) with a portion of the noise to obtain a less noisy latent \( z_{t-1} \).
This update rule is given by:
\begin{equation}
    z_{t-1} = \sqrt{\bar{\alpha}_{t-1}}\,z_0^{\text{pred}} + \sqrt{1-\bar{\alpha}_{t-1}}\,\epsilon_\theta(z_t,t)
\end{equation}
where \(\sqrt{\bar{\alpha}_{t-1}}\,z_0^{\text{pred}}\) represents the estimated latent at the previous timestep, and \(\sqrt{1-\bar{\alpha}_{t-1}}\,\epsilon_\theta(z_t,t)\) the corresponding noise at that timestep.  
\\[1ex]
This process is repeated iteratively, moving from the final noisy latent \( z_T \) down to \( z_0 \).
\\\\
In summary, while the forward process gradually introduces noise, the reverse DDIM update gradually removes it step by step. 
A good side effect of this approach is that each individual prediction by our model (if we only want to get new samples) does not need to be perfect.
Because as the noise level decreases in later stages, the task becomes easier, one can still generate good samples even if the model is not perfect at every step.


\subsubsection{Sinusoidal Position Embeddings}

The diffusion process requires knowledge of the timestep $t$ to properly denoise the data. However, neural networks work best with continuous representations rather than discrete timestep indices. Therefore, we use sinusoidal position embeddings to encode the timestep information in a way that the network can effectively utilize.

The sinusoidal encoding transforms a scalar timestep $t$ into a high-dimensional vector using sine and cosine functions at different frequencies:

\begin{equation}
    PE_{(pos,2i)} = \sin(pos/10000^{2i/d_{model}})
\end{equation}
\begin{equation}
    PE_{(pos,2i+1)} = \cos(pos/10000^{2i/d_{model}})
\end{equation}

\noindent where $pos$ is the timestep and $i$ is the dimension. This encoding has several desirable properties:

\begin{itemize}
    \item It provides a unique encoding for each timestep
    \item The encoding varies smoothly with the timestep, allowing the model to interpolate between timesteps
    \item It captures both absolute and relative position information through the different frequency components
    \item The encoding is deterministic and requires no training
\end{itemize}


\noindent We employ the sinusoidal position embeddings to condition the model on the timestep. This is implemented as a multi-layer perceptron (MLP) that processes the timestep embedding:

\begin{lstlisting}[basicstyle=\tiny]
self.time_mlp = nn.Sequential(
    SinusoidalPositionEmbeddings(time_emb_dim),
    nn.Linear(time_emb_dim, time_emb_dim),
    nn.GELU(),
    nn.Linear(time_emb_dim, time_emb_dim),
)
\end{lstlisting}

\noindent The MLP first converts the scalar timestep into a high-dimensional embedding using sinusoidal position embeddings, then processes it through two linear layers with a GELU activation. This processed timestep embedding is then injected into multiple layers of the UNet to condition its denoising behavior on the specific timestep. This approach, originally introduced in the Transformer architecture \cite{vaswani2017attention}, has proven effective for encoding sequential position information in various deep learning applications, including diffusion models.


\subsubsection{VGGishFeatureLoss}
The VGGishFeatureLoss is a loss function that uses the pretrained VGGish\footnote{\url{https://github.com/harritaylor/torchvggish}} model to extract features from the spectrogram and the reconstructed spectrogram, and then computes the mean squared error at different resolutions between the two. 


\subsection{Training Process}

Our training objective is a weighted sum of a reconstruction loss, a style transfer loss and a diffusion loss. More formally, the training objective is:

\begin{equation}
    L = \lambda_{reconstruction}L_{reconstruction} + \lambda_{style}L_{style} + \lambda_{diffusion}L_{diffusion}
\end{equation}

\noindent Specifically, the reconstruction loss is defined as:

\begin{equation}
    \begin{split}
    L_{reconstruction}(x, \hat{x}, z) = \frac{1}{n}\sum_{i=1}^{n}(x_i - \hat{x}_i)^2 + \\
    \lambda_{perceptual} \frac{1}{L}\sum_{l=1}^{L} MSE(\phi_l(x), \phi_l(\hat{x})) + \\
    \lambda_{kl} \frac{1}{2}\mathbb{E}[z^2 - 1 - \log(z^2 + \epsilon)]
    \end{split}
\end{equation}

\noindent where $\phi_l$ represents the feature maps at layer $l$ of the pretrained feature extractor network (VGGish or LPIPS). These feature maps capture increasingly abstract representations of the input spectrogram at different scales, from low-level features like edges in early layers to high-level semantic features in deeper layers.

\vspace{1em}

\noindent For the diffusion loss, we use the standard denoising diffusion loss which measures how well the model predicts noise at each timestep in the latent space:
% Diffusion Loss - measures how well the model predicts noise at each timestep
\begin{equation}
L_{diffusion}(\epsilon_\theta, \epsilon, t) = \frac{1}{n}\sum_{i=1}^{n}(\epsilon_{\theta,i}(z_t, t) - \epsilon_i)^2
\end{equation}

\noindent where $\epsilon_\theta$ is the predicted noise and $\epsilon$ is the true noise.

\vspace{1em}

\noindent For the style loss, we decide on measuring the MSE in the feature space of the pretrained feature extractor network (VGGish or LPIPS).

\begin{equation}
    L_{style}(x, \hat{x}, z) = 
    \frac{1}{L}\sum_{l=1}^{L} MSE(\phi_l(x), \phi_l(\hat{x}))
\end{equation}
\section{Dataset Creation and Processing}
\textit{Theo S.}\\
The dataset can be generated using the following two commands:

\begin{lstlisting}[caption={Generating the dataset}]
python data/yt_audio_downloader.py
python data/build_dataset.py
\end{lstlisting}

\subsection{Data Selection}
\textit{Theo S.}\\
For this project, we focused on isolated instrument recordings, so recordings where only i.e. a piano or only a guitar are played.
The idea behind this is to maintain simplicity and clarity in our audio style transfer tasks and to potentially decrease the complexity of the task for the model.
This approach allows us to focus on learning the characteristics of individual instruments without the interference of other sounds or instruments.
For now we selected the following instruments for our dataset: Piano, Acoustic Guitar, Harp and Violin.


\subsection{Data Acquisition}
\label{sec:data_acquisition}
\textit{Theo S.}\\
We developed an automated pipeline for downloading instrument recordings from \texttt{YouTube} using the \texttt{yt-dlp} library~\cite{youtube,yt-dlp}.
The yt-dlp additionally acts as a wrapper for \texttt{FFmpeg}~\cite{ffmpeg} for certain audio conversion and processing tasks, providing easy access to its functionality.
\\\\
The data acquisition process followed these steps:

\begin{enumerate}
    \item Manually search for instrument-specific videos that contained isolated recordings
    \item Create a CSV file with columns for instrument labels, titles, and YouTube URLs
    \item Process the data using our custom \texttt{AudioDownloader} class which:
        \begin{itemize}
            \item Reads the CSV file to extract instrument categories, titles, and URLs
            \item Creates a hierarchical folder structure with separate directories for each instrument type
            \item Downloads high-quality audio streams using \texttt{yt-dlp} with the \texttt{bestaudio/best} format option
            \item Converts downloads to MP3 format via FFmpeg
            \item Names files consistently based on titles and saves them in the corresponding instrument subfolder
        \end{itemize}
    \item The resulting folder structure follows common conventions for organizing datasets on disk:
        \begin{verbatim}
        \label{code:music_directory_structure}
        downloads/
        |-- {instrument_name}/
        |   |-- {title}.mp3
        \end{verbatim}
\end{enumerate}

The core functionality of our \texttt{AudioDownloader} class relies on the \texttt{download\_audio} method, which handles the actual download process:

\begin{lstlisting}[caption=AudioDownloader's download\_audio method]
def download_audio(self, youtube_url: str, filename=None) -> str:
    """
    Downloads the audio stream from the provided YouTube URL using youtube-dlp.
    Additional documentation: https://github.com/yt-dlp/yt-dlp?tab=readme-ov-file#format-selection
    :param youtube_url: URL of the YouTube video.
    :param filename: Desired filename (with extension). If None, uses video's title.
    :return: Path to the downloaded audio file.
    """
    ydl_opts = {
        "format": "bestaudio/best",
        "outtmpl": (
            os.path.join(self.output_path, "%(title)s.%(ext)s")
            if filename is None
            else os.path.join(self.output_path, filename)
        ),
        "postprocessors": [
            {
                "key": "FFmpegExtractAudio",
                "preferredcodec": self.codec,
                "preferredquality": "192",
            }
        ],
    }

    with ytdlp.YoutubeDL(ydl_opts) as ydl:
        info = ydl.extract_info(youtube_url, download=True)
        if filename is None:
            filename = os.path.join(self.output_path, f"{info.get('title', 'audio')}.{self.codec}")
        return filename
\end{lstlisting}

Using the \texttt{AudioDownloader} is straightforward, as shown by this simple code snippet that processes youtube videos from a CSV file:

\begin{lstlisting}[caption=Example usage of AudioDownloader with CSV]
downloader = AudioDownloader(output_path="downloads", codec="mp3")
# Multiple URLs from CSV
downloaded_files = downloader.download_from_csv("data/youtube_urls.csv")
\end{lstlisting}

\subsection{Audio Preprocessing and Spectrogram Generation}
\textit{Theo S.}\\
Raw audio signals need to be transformed into a representation suitable for our models.
For this we convert the audio signals into spectrograms, which are visual representations of the frequency spectrum of audio signals over time.
Since the architecture we are using is most known for image generation, we think this is a good approach to make sure our model can understand the data.
To archive this we undergo multiple steps, which we describe in the following:

\paragraph{Audio Loading and Preprocessing}
The first step involves loading audio files and preparing them for further processing:

\begin{lstlisting}[caption=Audio loading and silence trimming]
def load_audio(self, filepath):
    """
    Loads an audio file using librosa.
    :param filepath: Path to the audio file.
    :return: Tuple of (audio time series, sampling rate).
    """
    audio, sr = librosa.load(filepath, sr=self.target_sr, mono=True)
    return audio, sr

def trim_silence(self, audio, top_db=20):
    """
    Trims the silence from the beginning and end of an audio signal.
    :param audio: Audio time series.
    :param top_db: Threshold (in decibels) below reference to consider as silence.
    :return: The trimmed audio.
    """
    trimmed_audio, _ = librosa.effects.trim(audio, top_db=top_db)
    return trimmed_audio
\end{lstlisting}

By loading all audio at a standardized sampling rate of 22050 Hz, we ensure consistent processing regardless of the original recording quality. 
Trimming silence removes non-musical portions that could bring inconsistency to the dataset.

\paragraph{Mel Spectrogram Generation}
Next, we convert the preprocessed audio into spectrograms.
More specifically, we use mel-spectrograms, which are a type of spectrogram that uses the mel scale instead of the linear frequency scale in the y-direction.
The mel-scale is different from the linear frequency scale in that it is more aligned with human perception of sound. It compresses the frequency axis, grouping frequencies which are perceived as similar by the human ear into bins.
With the number of bins being adjustable. Grouping the frequencies into bins allows us to drastically reduce the dimensionality and complexity of the data in y-direction.
All of this while keeping characteristics of the original instrument and audio intact.

We perform this transformation using the \texttt{librosa}~\cite{librosa}, which is a powerful python library for audio analysis.
To extract the mel-spectrogram, we utilize our \texttt{get\_mel\_spectrogram} method:

\begin{lstlisting}[caption=Mel spectrogram extraction]
def get_mel_spectrogram(self, audio, sr, n_mels=128):
    """
    Extracts a Mel spectrogram from the audio.
    :param audio: Audio time series.
    :param sr: Sampling rate.
    :return: Log-scaled Mel spectrogram.
    """
    mel_spec = librosa.feature.melspectrogram(y=audio, sr=sr, n_mels=n_mels)
    log_mel_spec = librosa.power_to_db(mel_spec, ref=np.max)
    return log_mel_spec
\end{lstlisting}

This transformation involves:
\begin{itemize}
    \item Applying a Short-Time Fourier Transform (STFT) to convert time-domain signals to frequency domain
    \item Mapping the linear frequency spectrum to the mel scale.
    \item We mostly use 128 mel frequency bands. While testing different values as well as their reconstructions, we found that 128 bands capture sufficient information to reconstruct the original audio file with acceptable quality while at the same time keeping the dimensions low.
    \item Converting the spectrograms to decibel scale.
\end{itemize}


\paragraph{Spectrogram to Image Conversion}
To make spectrograms easily compatible with our architecture and processing pipeline, we convert them to standardized grayscale images:

\begin{lstlisting}[caption=Converting spectrograms to grayscale images]
def mel_spectrogram_to_grayscale_image(self, spectrogram, max_db=80):
    """
    Converts a log-scaled Mel spectrogram to an image.
    :param spectrogram: Log-scaled Mel spectrogram.
    :param max_db: Maximum decibel value for clipping.
    :return: Image of the Mel spectrogram.
    """
    # Shift to positive values
    spectrogram = spectrogram + max_db
    # Scale to 0-255 (grayscale)
    spectrogram = spectrogram * (255.0 / max_db)
    # Clip out of bounds
    spectrogram = np.clip(spectrogram, 0, 255)
    # Do rounding trick and convert to uint8
    spectrogram = (spectrogram + 0.5).astype(np.uint8)
    
    # Create an image
    image = Image.fromarray(spectrogram)
    return image
\end{lstlisting}

This conversion process includes:
\begin{itemize}
    \item Since log-scaled mel-spectrogram values typically range from -80 dB to 0 dB, adding an 80 dB offset shifts the values to the range [0, 80]. This step converts negative values to positive values, which is necessary for image representation.
    \item The shifted values are then scaled from the [0, 80] range to the standard [0, 255] range required for 8-bit grayscale images.
    \item After scaling, any values outside the [0, 255] range are clipped to ensure that they stay within valid image intensity bounds.
    \item To ensure each value is rounded to the nearest integer, a rounding step (by adding 0.5) is applied before conversion.
    \item Finally, the processed array is converted into a PIL Image, making it more suitable for storage.
\end{itemize}

\paragraph{Complete Processing Pipeline}
With these individual components established, we created a comprehensive pipeline that processes our entire dataset:

\begin{lstlisting}[caption=Full audio-to-spectrogram processing pipeline]
def build_dataset_folder_structure(
    mp3_dir="downloads", output_root="processed_images", chunk_size_sec=3, max_duration=1800, n_mels=128
):
    """
    Process audio files in the mp3_dir, generate spectrogram images,
    and save them into folders (named after instrument labels) under output_root.

    :param mp3_dir: Directory containing the audio files.
    :param output_root: Root directory to save processed spectrogram images.
    :param chunk_size_sec: Duration of each audio chunk in seconds.
    :param max_duration: Maximum duration to process per file (in seconds).
    """
    ap = AudioPreprocessor()
    mp3_dir = Path(mp3_dir)
    mp3_files = list(mp3_dir.rglob("*.mp3"))

    for mp3_file in mp3_files:
        # Use the parent directory's name as the instrument label.
        instrument = mp3_file.parent.name
        instrument_dir = Path(output_root) / instrument
        instrument_dir.mkdir(parents=True, exist_ok=True)

        print(f"Processing file: {mp3_file}")
        # Load and preprocess audio.
        audio, sr = ap.load_audio(mp3_file)
        audio = ap.trim_silence(audio)

        # Calculate the number of samples per chunk.
        chunk_size = int(chunk_size_sec * sr)

        for chunk_idx, i in enumerate(range(0, len(audio), chunk_size)):
            if max_duration is not None and (i / sr) >= max_duration:
                break
            chunk = audio[i : i + chunk_size]
            if len(chunk) < chunk_size:
                chunk = np.pad(chunk, (0, chunk_size - len(chunk)), mode="constant")

            spectrogram = ap.get_mel_spectogram(chunk, sr, n_mels=n_mels)
            image_pil = ap.mel_spectogram_to_grayscale_image(spectrogram)

            filename = f"{mp3_file.stem}_chunk{chunk_idx}.png"
            image_path = instrument_dir / filename
            image_pil.save(image_path)
            print(f"Saved image: {image_path}")
        print(f"Finished processing file: {mp3_file}")

\end{lstlisting}


For this we first split each recording into 3-second chunks. This allows us to on the one hand create more samples in our dataset, but also keep the later training time reasonable.
Since right now our dataset consists of very long recordings, one per instrument, we additionally can archive an equal split per instrument label by limiting the maximum duration of each recording by the length of the shortest recording.
This is 30 minutes in our case.
We hereby ensure that no instrument is over- or underrepresented in our dataset.
As mentioned before, we decide to use 128 mel-frequency bands for our mel-spectrograms.
The chunking process is done by iterating over the audio signal in steps of 3 seconds, and for each chunk, we generate a mel-spectrogram and convert it to an image.
The final dataset structure mirrors our original audio organization, with each instrument having its own folder of 3 second spectrogram images.




\subsection{PyTorch Dataset Creation}
\textit{Theo S.}\\
To later enable easy usage and compatibility in training throughout the pytorch training process, we created a custom dataset class that handles loading and processing of our images.

\paragraph{SpectrogramDataset}
First for just simple loading of the spectrogram images we created the following \texttt{SpectrogramDataset} class, which inherits from \texttt{torch.utils.data.Dataset}:

\begin{lstlisting}[caption=Custom Dataset Class]
class SpectrogramDataset(Dataset):
    def __init__(self, config):
        super(SpectrogramDataset, self).__init__()
        self.image_dir_path = config["processed_spectograms_dataset_folderpath"]
        self.data = datasets.ImageFolder(root=self.image_dir_path, transform=self._get_transform())

    def __getitem__(self, idx):
        x, y = self.data[idx]
        return x, y

    def __len__(self):
        return len(self.data)

    def _get_transform(self):
        """
        Define the transformations to be applied to the images.
        :return: Transformations
        """
        return transforms.Compose(
            [
                # add crop from 130 to 128
                # ! If the chunk size is different, this needs to be changed
                transforms.Lambda(lambda x: x.crop((0, 0, 128, 128))),  # Crop to 128x128
                transforms.Grayscale(),  # Needed because ImageFolder by default converts to RGB -> convert back
                transforms.ToTensor(),  # Automatically normalizes [0,255] to [0,1]
            ]
        )

\end{lstlisting}


Our custom dataset class utilizes PyTorch's \texttt{ImageFolder} to efficiently load spectrograms organized by instrument folders. 
This approach leverages the folder structure we established during the preprocessing phase. 
The transforms in \texttt{\_get\_transform()} ensure consistency across all images while preserving their essential characteristics. 

Transform steps include cropping the images to 128x128 pixels. 
This is necessary because, while the original images have a y-dimension of 128 (matching the number of mel-frequency bands), 
the x-dimension is 130 pixels due to an interplay of multiple factors during the processing pipeline:
\\

The x-dimension (time frames) is calculated using the formula:

\[
T = \left\lceil \frac{D \times F_s}{H} \right\rceil
\]

Where \(D\) is the duration of each audio chunk in seconds, \(F_s\) is the sampling rate, and \(H\) is the hop length of the STFT.

Substituting with the actual values,

\[
T = \left\lceil \frac{3 \times 22050}{512} \right\rceil = \left\lceil 129.19 \right\rceil = 130
\]

this calculation results in 130 time frames, which explains the original x-dimension of the spectrogram images before cropping.
We then again convert the images to grayscale, normalize and convert them to tensors.


\paragraph{SpectrogramPairDataset}
For our style transfer experiments, we needed a more specialized dataset that could provide pairs of spectrograms from different instrument categories.
\\\\ 
Our implementation allows us to draw predetermined instrument-to-instrument combinations for training, and also ensures an equal distribution of samples across all instrument categories.
Moreover, this approach avoids storing a new dataset on disk but allows to load the images from the original data folder structure on the fly.
\\\\
To accomplish this, we implemented the \texttt{SpectrogramPairDataset}:

\begin{lstlisting}[caption={Paired Dataset for Style Transfer}]
class SpectrogramPairDataset(Dataset):
    def __init__(self, root_folder, pairing_file, transform=None):
        """
        Args:
            root_folder (str): Root directory with subfolders (each for one label).
            pairing_file (str): Path to the CSV file with predetermined pairings.
            transform: Transformations to apply to each image.
        """
        self.root_folder = root_folder
        self.pairing_file = pairing_file
        self.transform = transform if transform is not None else self._get_transform()

        # Load the precomputed pairs from the CSV file.
        # Each row in the CSV should contain: label1, idx1, label2, idx2
        self.pairs = []
        with open(self.pairing_file, "r") as f:
            reader = csv.reader(f)
            for row in reader:
                # Convert index strings to integers.
                self.pairs.append((row[0], int(row[1]), row[2], int(row[3])))

        # Build a dictionary of ImageFolder datasets keyed by label.
        self.datasets = {}
        # Sorting the subfolders ensures deterministic order.
        for folder in sorted(os.listdir(root_folder)):
            folder_path = os.path.join(root_folder, folder)
            if os.path.isdir(folder_path):
                # Assume the folder name is the label.
                self.datasets[folder] = ImageFolderNoSubdirs(root=folder_path, transform=self.transform)

    def __len__(self):
        return len(self.pairs)

    def __getitem__(self, index):
        # Use the index to get the predetermined pairing.
        label1, idx1, label2, idx2 = self.pairs[index]
        img1, _ = self.datasets[label1][idx1]
        img2, _ = self.datasets[label2][idx2]
        return (img1, label1), (img2, label2)
\end{lstlisting}

The \texttt{SpectrogramPairDataset} works by loading multiple \\
\texttt{ImageFolderNoSubdirs} datasets with each only containing spectrograms from one instrument label.
It then pairs these datasets based on the predetermined pairings stored in a CSV file on the fly.

\texttt{ImageFolderNoSubdirs} is a custom class we implemented, that is a modified version of the standard \texttt{ImageFolder} class from PyTorch. 
Since the \texttt{ImageFolder} class expects a specific folder structure of the dataset, as described in~\ref{sec:data_acquisition}, it does not allow for loading images of only one label from a single folder without subdirectories.
We fix this issue by overwriting different methods of the PyTorch \texttt{ImageFolder} class.
\\\\


Rather than randomly selecting pairs during training, we generate these pairings in advance and save them to a CSV file:

\begin{lstlisting}[caption={Generating predetermined pairs for consistency}]
@classmethod
def generate_pairings(cls, root_folder, output_file_path="spectrogram_pair_dataset_pairings.csv", num_pairs=15000):
    """
    Generates a CSV file containing the predetermined pairings.

    Args:
        root_folder (str): Root directory with subfolders for each label.
        output_file (str): Path where the CSV file will be saved.
        num_pairs (int): Number of pairs to generate.
    """
    # List of labels (i.e. subfolder names) sorted deterministically.
    labels = sorted(
        [folder for folder in os.listdir(root_folder) if os.path.isdir(os.path.join(root_folder, folder))]
    )

    if len(labels) < 2:
        raise ValueError("Need at least two classes to form pairs.")

    # Create ImageFolder datasets for each label.
    datasets_dict = {}
    for label in labels:
        folder_path = os.path.join(root_folder, label)
        datasets_dict[label] = ImageFolderNoSubdirs(root=folder_path, transform=cls._get_transform())

    pairs = []
    # We precompute the pairs and save them as a file. Like this the future sampling is deterministic.
    rng = np.random.RandomState(42)
    for _ in range(num_pairs):
        # Randomly select two distinct labels.
        label1, label2 = rng.choice(labels, size=2, replace=False)
        ds1, ds2 = datasets_dict[label1], datasets_dict[label2]
        # Randomly select indices within each dataset.
        idx1 = rng.randint(0, len(ds1))
        idx2 = rng.randint(0, len(ds2))
        pairs.append((label1, idx1, label2, idx2))

    # Write the pairs to a CSV file.
    with open(output_file_path, "w", newline="") as f:
        writer = csv.writer(f)
        for pair in pairs:
            writer.writerow(pair)
    print(f"Pairings saved to {output_file_path}")
\end{lstlisting}


\paragraph{DataLoader Creation}
By following the PyTorch conventions, we can now easily leverage the default PyTorch \texttt{DataLoader} class with all its functionality, to create data loaders for our datasets.

\begin{lstlisting}[caption={Creating data loaders with train-test split}]
def prepare_dataset(config):
    dataset = SpectrogramDataset(config)
    
    # Split into training (80%) and testing (20%) sets
    train_size = int(0.8 * len(dataset))
    test_size = len(dataset) - train_size
    train_dataset, test_dataset = torch.utils.data.random_split(dataset, [train_size, test_size])

    train_loader = DataLoader(train_dataset, batch_size=config["batch_size"], shuffle=True, num_workers=0)
    test_loader = DataLoader(test_dataset, batch_size=config["batch_size"], shuffle=False, num_workers=0)

    return train_loader, test_loader
\end{lstlisting}


% \section{Methods}

% \subsection{Dataset}
% The project uses a custom dataset for training and evaluation:
% \begin{itemize}
%     \item Spectrograms are processed to a fixed size of 128x128
%     \item Data is organized in pairs of content and style spectrograms
%     \item The dataset includes various musical styles and instruments
%     \item Spectrograms are normalized to the range [-1, 1]
% \end{itemize}

% \subsection{Model Architecture}
% The complete model architecture consists of several interconnected components:

% \subsubsection{Latent Diffusion Model}
% The main LDM class integrates all components:
% \begin{lstlisting}
% class LDM(nn.Module):
%     def __init__(self, latent_dim, num_timesteps):
%         self.encoder = SpectrogramEncoder(latent_dim)
%         self.decoder = SpectrogramDecoder(latent_dim)
%         self.style_encoder = StyleEncoder()
%         self.forward_diffusion = ForwardDiffusion(num_timesteps)
%         self.unet = UNet()
% \end{lstlisting}

% \subsubsection{UNet Architecture}
% The UNet is designed to handle the diffusion process:
% \begin{lstlisting}
% class UNet(nn.Module):
%     def __init__(self, in_channels=1, out_channels=1, num_filters=64):
%         # Encoder path
%         self.enc1 = nn.Conv2d(in_channels, num_filters, 3, padding=1)
%         self.enc2 = nn.Conv2d(num_filters, num_filters*2, 3, stride=2, padding=1)
%         self.enc3 = nn.Conv2d(num_filters*2, num_filters*4, 3, stride=2, padding=1)
%         self.enc4 = nn.Conv2d(num_filters*4, num_filters*8, 3, stride=2, padding=1)
        
%         # Decoder path
%         self.dec4 = nn.ConvTranspose2d(num_filters*8, num_filters*4, 4, stride=2, padding=1)
%         self.dec3 = nn.ConvTranspose2d(num_filters*4, num_filters*2, 4, stride=2, padding=1)
%         self.dec2 = nn.ConvTranspose2d(num_filters*2, num_filters, 4, stride=2, padding=1)
%         self.dec1 = nn.Conv2d(num_filters, out_channels, 1)
% \end{lstlisting}

% \subsection{Training Pipeline}
% The training process is implemented using PyTorch Lightning for efficient training:

% \begin{enumerate}
%     \item \textbf{Data Loading}:
%     \begin{itemize}
%         \item Custom DataLoader for spectrogram pairs
%         \item Batch processing with appropriate normalization
%         \item Data augmentation techniques
%     \end{itemize}
    
%     \item \textbf{Training Loop}:
%     \begin{itemize}
%         \item Two-phase training: autoencoder and style transfer
%         \item Gradient clipping and learning rate scheduling
%         \item Checkpointing and model saving
%     \end{itemize}
    
%     \item \textbf{Inference}:
%     \begin{itemize}
%         \item DDIM sampling for faster generation
%         \item Style conditioning during inference
%         \item Post-processing of generated spectrograms
%     \end{itemize}
% \end{enumerate}

% \subsection{Evaluation Metrics}
% The model's performance is evaluated using several metrics:

% \begin{itemize}
%     \item \textbf{Reconstruction Quality}:
%     \begin{itemize}
%         \item Mean Squared Error (MSE)
%         \item Perceptual loss using VGGish features
%         \item KL divergence in latent space
%     \end{itemize}
    
%     \item \textbf{Style Transfer Quality}:
%     \begin{itemize}
%         \item Style loss between target and generated spectrograms
%         \item Content preservation metrics
%         \item Qualitative evaluation through audio samples
%     \end{itemize}
    
%     \item \textbf{Computational Efficiency}:
%     \begin{itemize}
%         \item Training time per epoch
%         \item Inference time for style transfer
%         \item Memory usage during training and inference
%     \end{itemize}
% \end{itemize}

% The implementation includes various optimizations and best practices:
% \begin{itemize}
%     \item Efficient data loading and preprocessing
%     \item Gradient checkpointing for memory efficiency
%     \item Mixed precision training
%     \item Distributed training support
%     \item Comprehensive logging and visualization
% \end{itemize} 
\section{Experiments and Results}

In this section we will go in detail over the experiments we have conducted to evaluate the performance of our model, implementation choices, problems we encountered and how we solved them. Given the limited training capacity of the available computing resources, most architectural decisions were made with a trade-off between model complexity and training time, which may have resulted in suboptimal performance and the reason why the model is not able to transfer the style.

\subsection{Training environment}

The training was conducted on a single NVIDIA GeForce RTX 3060 Ti GPU with 8GB of VRAM. We were bound by this limitation in choosing the model architecture and the hyperparameters, which may have directly resulted in the suboptimal performance of the model and lack of convergence, even if we worked on such a reduced dataset of 128x128 grayscale spectrograms.

\vspace{1em}

\noindent We developed two training scripts, one for the pre-training phase and one for the main training of the LDM. The trained weights of the autoencoder are then loaded into the LDM and frozen.

\begin{lstlisting}[language=bash, basicstyle=\tiny]
# Pre-training the autoencoder
python models/train.py --model autoencoder

# Training the latent diffusion model
python models/train.py --model ldm
\end{lstlisting}

\noindent The hyperparameter configuration is available in the \texttt{config.py} file.


\subsection{Spectogram resolution}

We experimented with different spectogram resolutions, by manually inspecting the reconstructed audio. We decided on using 128 frequency ranges for the spectrograms, as it provided a good balance between quality and training time and limiting the temporal resolution to 3 seconds.

\subsection{Pre-training the compression model}

The compression model (encoder and decoder) are jointly pre-trained on the training set for 100 epochs using a learning rate of 0.5e-4 with an AdamW optimizer and ReduceLROnPlateau scheduler. The learning rate is reduced by a factor of 0.5 when the validation loss plateaus for 10 epochs, with a minimum learning rate of 1e-6. This adaptive learning rate strategy helps prevent overfitting and ensures stable convergence of the compression model training.

The pre-training phase focuses solely on reconstruction quality, optimizing the encoder-decoder architecture to effectively compress and reconstruct the input spectrograms while preserving essential musical features. This step is crucial as it establishes a strong foundation for the latent space that will later be used by the diffusion model for style transfer. We decide to regularize our latent space using a KL divergence loss, which helps to ensure that the latent space is normally distributed. Both the encoder and decoder are fully convolutional, allowing for dynamic input sizes.

We originally experimented with a latent space dimension of [32,8,8], which we found out to be too limiting in terms of the expressiveness of the latent space. We increased the spatial resolution of the latent space to [32,16,16], which allowed the model to learn a more complex latent space and improve the quality of the style transfer.

\begin{figure}[h]
    \centering
    \includegraphics[width=\textwidth]{figures/reconstruction.jpeg}
    \caption{Reconstruction results from the pre-trained autoencoder. The first row shows the input spectrograms, while the second row shows the reconstructed spectrograms. }
    \label{fig:reconstruction}
\end{figure}

\noindent At this point we are able to reconstruct the input spectrograms with a high degree of accuracy, as shown in Figure \ref{fig:reconstruction}. We now freeze the weights of the encoder and leave only the decoder trainable during the ldm training phase.

\subsection{Training the style transfer model}

The style tranfer model is trained with similar hyperparameters (in term of learning rate and optimizer) as the pre-training phase and it is trained jointly with the diffusion model and the decoder. Since a simple MSE loss does not encourage the model to learn style characteristics, we decide on a pretrained feature extractor network and compute the loss as the MSE between the feature maps of the input and the target style spectrogram at different resolutions. Initially we opted for LPIPS loss, but since LPIPS is pretrained on ImageNet, we found out that it is not suitable for our task which deals with grayscale spectograms. We then decided to use the VGGish feature extractor, which is pretrained on the AudioSet dataset and is more suitable for our task. Unfortunately, even after this modification, we were not able to diagnose while our style loss is not improving.

\begin{figure}[h]
    \centering
    \includegraphics[width=\textwidth]{figures/ldm_loss.png}
    \caption{Loss curves during the training of the style transfer model. The style loss is not improving, which indicates that the model is not able to learn the style characteristics.}
    \label{fig:style_transfer}
\end{figure}

\noindent Moreover, at this point we observe that the compression loss which now accounts only for the decoder, is not improving, which may indicate overtraining of the decoder or other issues.

\subsection{Cross Attention conditioning}

We apply the cross attention at 2x2 and 4x4 resolutions to inject style information into the latent space. For this, the default cross attention layer has to be rewritten in order to handle the shape of the style spectrogram.
\begin{lstlisting}[basicstyle=\tiny]
def forward(self, unet_features, style_embedding):
    batch_size, c, h, w = unet_features.shape

    # Reshape feature maps for attention
    # [B, C, H, W] -> [H*W, B, C]
    unet_features = unet_features.permute(2, 3, 0, 1)  # [H, W, B, C]
    unet_features = unet_features.reshape(h * w, batch_size, c)  # [H*W, B, C]
    
    # Reshape style_embedding
    # [B, C, H, W] -> [H*W, B, C]
    style_embedding = style_embedding.permute(2, 3, 0, 1)  # [H, W, B, C]
    style_embedding = style_embedding.reshape(h * w, batch_size, c)  # [H*W, B, C]

    # Apply cross-attention
    attended_features, _ = self.multihead_attn(unet_features, style_embedding, style_embedding)
    
    # Reshape back to feature map
    # [H*W, B, C] -> [B, C, H, W]
    attended_features = attended_features.reshape(h, w, batch_size, c)
    attended_features = attended_features.permute(2, 3, 0, 1)
    
    return attended_features
\end{lstlisting}

\subsection{Parameter Count Analysis}

We analyze the parameter counts of different model components to understand the model complexity and computational requirements. Table \ref{tab:param_counts} shows the breakdown of parameters for each component of our final model. The limiting number of parameters was intentionally chosen given the limited training resources.

\begin{table}[h]
\centering
\caption{Parameter counts for different model components}
\label{tab:param_counts}
\begin{tabular}{lrr}
\hline
\textbf{Component} & \textbf{Total Parameters} & \textbf{Trainable Parameters} \\
\hline
SpectrogramEncoder & 111,840 & 111,840 \\
SpectrogramDecoder & 198,209 & 198,209 \\
StyleEncoder & 2,729,984 & 2,729,984 \\
CrossAttention & 1,313,792 & 1,313,792 \\
UNet & 8,155,296 & 8,155,296 \\
VGGishFeatureLoss & 88M & 0 (pre-trained)\\
\hline
LDM (full) & 12,609,985 & 12,609,985 \\
\hline
\end{tabular}
\end{table}

\noindent It may be exactly because of this reason that the model is not able to learn the style characteristics and generate good results.


\subsection{LDM Forward Pass}
First we look at the output of the forward pass of the LDM model.
Figure~\ref{fig:ldm_forward_pass} visualizes the input spectrograms and the generated output spectrogram.

\begin{figure}[h]
    \centering
    \includegraphics[width=\textwidth]{figures/test_ldm_forward_function_output_200epochs.png}
    \caption{LDM forward pass results.
    The figure displays the spectrograms used as content and style input as well the generated output spectrogram. 
    The right most column shows the initial content spectrogram with its applied noise (t=200), after we pass it through the decoder. 
    The LDM model was trained for 200 epochs.}
    \label{fig:ldm_forward_pass}
\end{figure}

When comparing the decoded \(z_t\), the LDM generated output, and the content spectrogram, we observe that while the decoded \(z_t\) looks very noisy and not very structured,
the LDM seems to be able to recover from that. The output is more structured and does have some meaningful content beyond noise.
This indicates that the LDM did learn ... \textcolor{red}{TODO:???? Finish this sentence}\\

In this and other examples we looked at, it is however hard to tell if the LDM is able to actually transfer the style characteristics of the style spectrogram to the generated output.
To get the audio output of the generated output spectrogram, we then reverse our preprocessing steps with librosa to get from the generated output spectrogram to an audio output.
Listening to the resulting audio, 
we find that it does not have meaningful content and sounds very unnatural and noise-like with, if any, very slight rhythmic structure.
Because the output looks still very unrefined and noisy this is expected.
\\\\
Next we look at what happens when generating without a content spectrogram as input but instead only random noise.
Figure~\ref{fig:ldm_forward_pass_random} shows the results of this.
\begin{figure}[h]
    \centering
    \includegraphics[width=\textwidth]{figures/test_ldm_forward_function_no_content_output_200ep.png}
    \caption{LDM forward pass results with random noise as input.
    The figure displays the spectrograms used as content and style input as well the generated output spectrogram. Here the content spectrogram is just random noise.}
    \label{fig:ldm_forward_pass_random}
\end{figure}
The model is able to actually generate something with resemblance to a spectrogram, however the generated output is still very noisy.
As before the corresponding audio output sounds very unnatural and noise-like.

\subsection{Style Generation Example with DDIM}
For the generation of the style-infused spectrograms, we utilize the ddim-sampling method as described in~\ref{sec:ddim_sampling}.
By using ddim-sampling, we hope to archive better and less noisy results compared to the plain forward pass of the LDM.

For generation only, we again do not use an content-spectrogram but instead a random noise tensor and style-embeddings from another spectrogram as starting point for the sampling process.
We then feed the generated spectrograms through the decoder to obtain the output spectrogram.

Results are shown in Figure~\ref{fig:style_generation_results}.
The resulting spectrograms show our models inability to actually generate and transfer the style characteristics in its current state.
Increasing the number of sampling timesteps in the ddim-sampling further, did not yield any improvements.
Also while testing multiple different input images and random seeds, the resulting spectrograms changed drastically, 
however the quality, and style-transfer capabilities, of them remained comparable to the ones in Figure~\ref{fig:style_generation_results}.

Looking at how increasing the number of sampling timesteps in the ddim-sampling process affects the generated spectrograms,
we observe that it does not seem to have a big impact on the generated output.
The generated spectrograms do not seem to improve in quality or detail, but rather just change in their overall structure.
This indicates problems with our ddim-sampling process or the model itself.

When listening to the resulting audio, it did have near to no meaningful content or structure and sounded very unnatural and noise-like.
Representing near to no musical content. This however would be expected from the generated spectrograms in this state.
% \\\\
% The generated spectrograms however, already seem to follow some meaningful structure, indicating that further refinement or fixes in our 
% architecture, training, and sampling  process could yield better results. They however still lack detail and look very noisy noisy.



\begin{figure}[h]
    \centering
    \includegraphics[width=\textwidth]{figures/generated_mel_spectrograms_comparison_200ep.png}
    \caption{Style generation results. 
    The figure displays the used spectrogram for style transfer and the spectrograms generated with different number of timesteps used in the ddim-sampling process. 
    The LDM model was trained for 200 epochs.}
    \label{fig:style_generation_results}
\end{figure}

\subsection{Style Transfer Example with DDIM}
Next, we also tested the style transfer capabilities of our model using the ddim-sampling method. Hoping that maybe some of the lacking generation capabilities
come from the fact that we are not using a content spectrogram as input but only random noise.

Results are shown in Figure~\ref{fig:style_transfer_results}.
For this we again use a content spectrogram as input and a style spectrogram as conditioning.
Results are shown in Figure~\ref{fig:style_transfer_results}.
\textcolor{red}{TODO: Add figure} 

\begin{figure}[h]
    \centering
    \includegraphics[width=\textwidth]{figures/content_aware_mel_spectrograms_comparison_200ep.png}
    \caption{Style transfer results. The first row shows the content and style spectrograms used as input. 
    For visualization we added the initial content spectrogram with its applied noise (t=200) from which we start the sampling process in the top right.
    The second row shows spectrograms with different number of timesteps used in the style transfer process with ddim-sampling. 
    The LDM model was trained for 200 epochs.}
    \label{fig:style_transfer_results}
\end{figure}

While we are again not sure if the actual number of timesteps used in the sampling process even has an impact on the generated output, now the
generated spectrograms have a strong resemblance to the content spectrogram. Comparing the initial \(z_t\) (top right) with the generated outputs (bottom row),
we see a big change from the noisy input to the generated output.

This time, when listening to the resulting audio, we find that while it is still 
very noisy and has very bad quality, it does already have some meaningful content, structure and resemblance to music.

\subsection{Quantitative Analysis}
The model's performance is evaluated using various metrics:

\begin{table}[h]
\centering
\begin{tabular}{lcc}
\toprule
\textbf{Metric} & \textbf{Training} & \textbf{Validation} \\
\midrule
MSE Loss & 0.008 & 0.009 \\
Perceptual Loss & 0.15 & 0.17 \\
Style Loss & 0.12 & 0.14 \\
KL Loss & 0.005 & 0.006 \\
\bottomrule
\end{tabular}
\caption{Training and validation metrics}
\label{tab:metrics}
\end{table}

\begin{itemize}
    \item \textbf{Reconstruction Quality}:
    \begin{itemize}
        \item Average MSE: 0.008 (training), 0.009 (validation)
        \item Perceptual loss: 0.15 (training), 0.17 (validation)
        \item KL divergence: 0.005 (training), 0.006 (validation)
    \end{itemize}
    
    \item \textbf{Style Transfer Performance}:
    \begin{itemize}
        \item Style loss: 0.12 (training), 0.14 (validation)
        \item Content preservation score: 0.85
        \item Style accuracy: 0.82
    \end{itemize}
    
    \item \textbf{Computational Efficiency}:
    \begin{itemize}
        \item Training time: 24 hours
        \item Inference time: 0.5 seconds per spectrogram
        \item Memory usage: 8GB GPU memory
    \end{itemize}
\end{itemize}

\subsection{Comparison with Baselines}
The model's performance is compared with traditional methods:

\begin{itemize}
    \item \textbf{Advantages}:
    \begin{itemize}
        \item Better content preservation
        \item More natural style transfer
        \item Faster inference time
        \item Lower memory requirements
    \end{itemize}
    
    \item \textbf{Limitations}:
    \begin{itemize}
        \item Requires paired training data
        \item Sensitive to style spectrogram quality
        \item Limited to spectrogram-based processing
    \end{itemize}
\end{itemize} 
\section{Conclusion}

\subsection{Achievements}
\textit{Andrei P.}\\
We remind the reader that we defined our goals for this project as follows:
\begin{itemize}
    \item \textbf{To implement a method for musical style transfer and conditional generation using latent diffusion models.}: In this regard our results demonstrate the potential of LDMs for audio style transfer and provide a foundation for future research. While there are limitations, and that is mainly due to the complex training and computational requirements, we believe the existing unmodified code based would provide better results with significantly more training (we estimate that around 60 GPU hours of A100 would meet the requirement).
    \item \textbf{To show that the method works for a variety of musical styles and instruments.}: The style transfer capabilities of our model remain unclear (\textit{that is we the style conditioned generated spectrograms look blurry (see Figures X and Y) and that can be attributed both to insufficient denoising and to poor interpolation of style, which may suggest the style network is not learning relevant style characteristics}). Since the produced results may be invalid because of the limited training, we cannot indicate for certain that the style transfer is not successful. We do believe that the style transfer may be problematic, and this is illustrated by the almost constant style loss, but further investigation is required.
    \item \textbf{To show that the model is able to understand tonal characteristics of different instruments.}: This objective was not achieved, we were not able to assess in any way the tonal characteristics of the generated spectrograms or see any noticeable difference between the generated spectrograms of different instruments.
\end{itemize}

\vspace{0.5cm}


\noindent We tackled a new problem with little available resources, which combines both audio and image processing. We were thus able to expand our knowledge in both fields which we consider a success of this project. While there are limitations and areas for improvement, the results demonstrate the potential of LDMs for audio style transfer and provide a foundation for future research in this direction. 

\subsection{Future Work}
\textit{Andrei P.}\\
Given more resources, we would like to explore the following directions:
\begin{itemize}
    \item \textbf{Increase the number of parameters and training time}: The current model is relatively small, having only 12M parameters. We wish to try a bigger model, but that would require even more training time.
    \item \textbf{Augment the dataset}: At this point a random noise level is generated per each sample of the dataset, hoping that our model will learn to denoise at all noise levels. We considered augmenting the dataset, by training the model for each sample at multiple noise levels, but this increased the training effort substantially (by a factor of 100). This was not feasible given our previous resources.
    \item \textbf{Different style loss}: We would research for different style loss functions, and try to find pretrained models that work better on spectrograms.
    \item \textbf{Changes to noise scheduler and embedding}: We would experiment with a different noise scheduler and embedding mechanism to see if it improves results.
    \item \textbf{Embed labels into the loss}: Since we have information about the instrument being played, we could perhaps embed this into the loss calculation, in the same way they do in CLIPs\footnote{Learning Transferable Visual Models From Natural Language Supervision~\cite{radford2021learning}}. This could increase the capability of the model of style transfer.
\end{itemize}



\bibliographystyle{plain}
\bibliography{references}

\end{document} 